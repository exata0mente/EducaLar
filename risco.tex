\chapter[Riscos]{Riscos}

Durante o encaminhamento do projeto, ter ciência de que está inviável prosseguir com o investimento proposto inicialmente, devido a atrasos de entrega dos resultados no tempo estipulado gerando mais carga horária dos funcionários, ou devido a troca de ferramentas por defeito ou danos (descarga elétrica ou uso indevido, por exemplo).

Ter desentendimentos de ideias referentes a maneira que se deve conduzir o projeto para ter um melhor desempenho, ou ter baixo rendimento de equipe por conta de falta animo ou desinteresse, além de não se ter uma liderança competente para fazer uma gestão eficiente de tempo e recursos.

As ferramentas usadas para o desenvolvimento do projeto devem estar a altura do desafio elas também definem o tempo que se gastara para executar o projeto, para nenhum desfalque por conta disso e manter sempre um ritmo de progresso constante.

Estar preparado para qualquer problema externo como \textit{blackout}, falta de funcionário por um ou mais dias, podendo ser mais de um, por qualquer que seja o motivo da ausência (afinal estão todos ganhando por hora trabalhada).
% Professora quer mais detalhes em cada tópico