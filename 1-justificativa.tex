\chapter[Justificativa]{Justificativa}

Com a massificação de celulares, uma grande quantidade de aplicativos \emph{mobile} é desenvolvida diariamente. Há de diversos tipos: produtividade, bem-estar, relacionamento, aprendizagem, vídeo, música, notícias, redes sociais, fotos, e mais uma imensidão de opções. 
Com isso é muito difícil encontrar uma pessoa que não possua um aplicativo que o auxilie em algo. 

Um dos aplicativos mais famosos, por exemplo, é o Tinder que possibilita o encontro de pessoas com interesses em comum, voltado a relacionamentos pessoais. O Tinder cruza algumas informações e apresenta opções ao usuário classificar se gostou ou não de determinada pessoa. Havendo uma resposta positiva (\emph{match}) e recíproca, essas duas pessoas podem estabelecer contato e então iniciar um relacionamento.

Outro aplicativo muito popular é o Uber, que permite ao usuário solicitar uma viagem pelo aplicativo onde o motorista é também um usuário que utiliza seu carro para executar a viagem. Estes aplicativos conectam dois lados de usuários que tem interesses em comum.

Pensando neste tipo de serviço identificamos que é possível disponibilizar uma plataforma em que pessoas que possuam bons conhecimentos em determinado assunto possam auxiliar outras que possuam dificuldades neste mesmo assunto, fornecendo assim um reforço educacional. 

Um exemplo de aplicação seria: um aluno do ensino médio possue dificuldades nas matérias de exatas e precisa de reforço para conseguir melhorar suas notas. Do outro lado há um estudante de letras que tem facilidade e dá aulas aos finais de semanas para seu primo. O aplicativo poderá unir estas duas pessoas em um bem comum.

% Separar em mais parágrafros