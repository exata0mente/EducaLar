\chapter[Cronograma]{Cronograma}

\emph{Nota dos Autores: Aqui será elencado na tabela os tópicos que estão em \textbf{Metas} que está pendente com o Caio}

% Como estamos falando de um aplicativo de médio porte, padronizando horário com aplicativos criados ele demoraria de 500 a 800 horas para ser desenvolvido. 

A ideia sugere um aplicativo de médio porte, o que pode levar de 500 a 800 horas para ser desenvolvido, projetando um \textit{lead time} de 5 a 9 meses.

%---- Aqui vai uma tabela ----


\begin{table}[htb]
  \centering
  \caption[Cronograma]{Cronograma}
  \label{tabCrono}
  \begin{tabular}{llllllll}
    \textbf{Atividades}                    		& \textbf{Jun} & \textbf{Jul} & \textbf{Ago} & \textbf{Set} & \textbf{Out} & \textbf{Nov} & \textbf{Dez} \\
    \hline
    Meta 1                 		& X   & X   & X   & X   & X   & X   & X   \\
    Meta 2  		& X   &     &     &     &     &     &     \\
    Meta 3                  		& X   &     & X   &     & X   &     & X   \\
    Meta 4 	&     & X   & X   &     &     &     &     \\
    Meta 5 	&     & X   & X   & X   & X   & X   &     \\
    Meta 6 			&     & X   & X   & X   & X   & X   & X   \\
    Meta 7 			&     &     &     &     &     & X   & X   \\
    Meta 8 			&     &     &     &     &     & X   & X   \\
    Meta 9 			&     &     &     &     &     &     & X   \\
    Meta 10 		&     &     &     &     &     &     & X   \\
    \hline
  \end{tabular}
\end{table}


% O projeto em si teria uma média de 5 a 9 meses de execução, mesmo ele tendo um tempo razoável de desenvolvimento, não é possível resolver nesse período, pois as etapas exigem diferentes pessoas e isso retarda o desenvolvimento. 
%
%
% Devemos colocar em foco que esse período estimado seja de desenvolvimento e lançamento do aplicativo, sua manutenção não tem como ser estimada em tempo nem custo (para o caso falhas ou bugs). 
%
%
% Não está legal. Montar um cronograma \textit{like a} iniciação científica associando o tópico de metas
